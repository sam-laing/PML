\documentclass[12pt]{article}

\usepackage{mathptmx}
\usepackage{bbm}
\usepackage{tikz-cd,mathtools}
\usepackage{tikz}
\usepackage{mathtools}
\usepackage{array}
\usepackage[utf8]{inputenc}
\usepackage[T1]{fontenc}
\usepackage{textcomp}
\usepackage[english]{babel}
\usepackage{amsmath, amssymb}
\usepackage[mathscr]{euscript}
\usepackage{subcaption}
\usepackage[margin=1in]{geometry}
\usepackage{graphicx}
\usepackage{listings}
\usepackage{xcolor}
\graphicspath{ {~/Maths/3rd_sem1/} }
\newtheorem{theorem}{Theorem}[section]
\newtheorem{lemma}[theorem]{Lemma}
\newtheorem{proposition}[theorem]{Proposition}
\newtheorem{corollary}[theorem]{Corollary}

\usepackage{hyperref}
\hypersetup{
	colorlinks=true,
        linkcolor=blue,
        filecolor=magenta,
        urlcolor=cyan,
}
\urlstyle{same}

\newcommand{\A}{\mathbb{A}}
\newcommand{\N}{\mathbb{N}}
\newcommand{\Z}{\mathbb{Z}}
\newcommand{\Q}{\mathbb{Q}}
\newcommand{\R}{\mathbb{R}}
\newcommand{\C}{\mathbb{C}}
\newcommand{\f}{\mathfrak{f}}
\newcommand{\F}{\mathbb{F}}
\newcommand{\g}{\mathfrak{g}}
\newcommand{\K}{\mathbb{K}}
\renewcommand{\l}{\mathfrak{l}}
\newcommand{\p}{\mathfrak{p}}
\renewcommand{\P}{\mathfrak{P}}
\newcommand{\PP}{\mathbb{P}}



\newenvironment{proof}[1][Proof]{\begin{trivlist}
\item[\hskip \labelsep {\bfseries #1}]}{\end{trivlist}}
\newenvironment{definition}[1][Definition]{\begin{trivlist}
\item[\hskip \labelsep {\bfseries #1}]}{\end{trivlist}}
\newenvironment{example}[1][Example]{\begin{trivlist}
\item[\hskip \labelsep {\bfseries #1}]}{\end{trivlist}}
\newenvironment{remark}[1][Remark]{\begin{trivlist}
\item[\hskip \labelsep {\bfseries #1}]}{\end{trivlist}}

\newcommand{\qed}{\nobreak \ifvmode \relax \else
      \ifdim\lastskip<1.5em \hskip-\lastskip
      \hskip1.5em plus0em minus0.5em \fi \nobreak
      \vrule height0.75em width0.5em depth0.25em\fi}

% figure support
\usepackage{import}
\usepackage{xifthen}
\pdfminorversion=7
\usepackage{pdfpages}
\usepackage{transparent}
\newcommand{\incfig}[1]{%
\def\svgwidth{\columnwidth}
\import{./figures/}{#1.pdf_tex}
}


\pdfsuppresswarningpagegroup=1

\begin{document}
\begin{center}
\Huge{\boldmath{PML HW1}}
\end{center}
\[
	\text{Submitted by Sam, Laing: 6283670 and Albert Catalan Tatjer 6443478} 
\] 
\subsection*{a)}
Assume $P(B|A) \ge P(B)$\\
We have
\begin{align*}
	P(B) &= P(B|A)P(A)+P(B|\neg A)(1-P(A)) \quad (\text{By partition law})\\
	     &\ge P(B)P(A) + P(B|\neg A)[1-P(A)]  
\end{align*}
\begin{align*}
	\implies \ P(B) [ 1-P(A)] \ge  P(B|\neg A) [1-P(A)] \implies P(B) \ge P(B|\neg A)
\end{align*}
\subsection*{b)} 
\[
	P(A|B) = \frac{P(A,B)}{P(B)} = \frac{P(B|A)P(A)}{P(B)} \ge  \frac{P(B)P(A)}{P(B)} = P(A)
\] 
\subsection*{c)} 
We have 
\begin{align*}
	P(A) &= P(A|B)P(B) + P(A|\neg B)(1-P(B)) \\
	     &\ge P(A)P(B) + P(A|\neg B)(1-P(B)) \quad \text{(from (b))}
\end{align*}
\begin{align*}
	\implies P(A)[1-P(B)] \ge  P(A|\neg B)[1-P(B)] \implies P(A) \ge P(A|\neg B)
\end{align*}
\\
\subsection*{d)} 
Now we assume that $P(B|A) = 1$:\\
We first observe that this immediately implies that $P(\neg B|A) = 0$.
Then we have:
 \begin{align*}
	 P(\neg A|\neg B) = \frac{P(\neg A,\neg B)}{P(\neg B)} = \frac{P(\neg A,\neg B)}{P(\neg B,A) + P(\neg B,\neg A)} 
\end{align*}
But we observe that now $P(\neg B,A) = P(\neg B|A)P(A) = 0$. Therefore our expression reduces to 1
\subsection*{e)} 
This is just a special case of part (a) (since $1 = P(B|A) \ge P(B)$ with $P(B) \in [0,1]$ by definition of probability)
\subsection*{f)} 
This is just a special case of part (c) (since $1 = P(A|B) \ge P(A)$ necessarily)
\end{document}
